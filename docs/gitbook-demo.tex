% Options for packages loaded elsewhere
\PassOptionsToPackage{unicode}{hyperref}
\PassOptionsToPackage{hyphens}{url}
%
\documentclass[
  oneside]{book}
\usepackage{amsmath,amssymb}
\usepackage{lmodern}
\usepackage{ifxetex,ifluatex}
\ifnum 0\ifxetex 1\fi\ifluatex 1\fi=0 % if pdftex
  \usepackage[T1]{fontenc}
  \usepackage[utf8]{inputenc}
  \usepackage{textcomp} % provide euro and other symbols
\else % if luatex or xetex
  \usepackage{unicode-math}
  \defaultfontfeatures{Scale=MatchLowercase}
  \defaultfontfeatures[\rmfamily]{Ligatures=TeX,Scale=1}
\fi
% Use upquote if available, for straight quotes in verbatim environments
\IfFileExists{upquote.sty}{\usepackage{upquote}}{}
\IfFileExists{microtype.sty}{% use microtype if available
  \usepackage[]{microtype}
  \UseMicrotypeSet[protrusion]{basicmath} % disable protrusion for tt fonts
}{}
\makeatletter
\@ifundefined{KOMAClassName}{% if non-KOMA class
  \IfFileExists{parskip.sty}{%
    \usepackage{parskip}
  }{% else
    \setlength{\parindent}{0pt}
    \setlength{\parskip}{6pt plus 2pt minus 1pt}}
}{% if KOMA class
  \KOMAoptions{parskip=half}}
\makeatother
\usepackage{xcolor}
\IfFileExists{xurl.sty}{\usepackage{xurl}}{} % add URL line breaks if available
\IfFileExists{bookmark.sty}{\usepackage{bookmark}}{\usepackage{hyperref}}
\hypersetup{
  pdftitle={Master-Thesis},
  pdfauthor={Axel Roth},
  hidelinks,
  pdfcreator={LaTeX via pandoc}}
\urlstyle{same} % disable monospaced font for URLs
\usepackage{color}
\usepackage{fancyvrb}
\newcommand{\VerbBar}{|}
\newcommand{\VERB}{\Verb[commandchars=\\\{\}]}
\DefineVerbatimEnvironment{Highlighting}{Verbatim}{commandchars=\\\{\}}
% Add ',fontsize=\small' for more characters per line
\usepackage{framed}
\definecolor{shadecolor}{RGB}{248,248,248}
\newenvironment{Shaded}{\begin{snugshade}}{\end{snugshade}}
\newcommand{\AlertTok}[1]{\textcolor[rgb]{0.94,0.16,0.16}{#1}}
\newcommand{\AnnotationTok}[1]{\textcolor[rgb]{0.56,0.35,0.01}{\textbf{\textit{#1}}}}
\newcommand{\AttributeTok}[1]{\textcolor[rgb]{0.77,0.63,0.00}{#1}}
\newcommand{\BaseNTok}[1]{\textcolor[rgb]{0.00,0.00,0.81}{#1}}
\newcommand{\BuiltInTok}[1]{#1}
\newcommand{\CharTok}[1]{\textcolor[rgb]{0.31,0.60,0.02}{#1}}
\newcommand{\CommentTok}[1]{\textcolor[rgb]{0.56,0.35,0.01}{\textit{#1}}}
\newcommand{\CommentVarTok}[1]{\textcolor[rgb]{0.56,0.35,0.01}{\textbf{\textit{#1}}}}
\newcommand{\ConstantTok}[1]{\textcolor[rgb]{0.00,0.00,0.00}{#1}}
\newcommand{\ControlFlowTok}[1]{\textcolor[rgb]{0.13,0.29,0.53}{\textbf{#1}}}
\newcommand{\DataTypeTok}[1]{\textcolor[rgb]{0.13,0.29,0.53}{#1}}
\newcommand{\DecValTok}[1]{\textcolor[rgb]{0.00,0.00,0.81}{#1}}
\newcommand{\DocumentationTok}[1]{\textcolor[rgb]{0.56,0.35,0.01}{\textbf{\textit{#1}}}}
\newcommand{\ErrorTok}[1]{\textcolor[rgb]{0.64,0.00,0.00}{\textbf{#1}}}
\newcommand{\ExtensionTok}[1]{#1}
\newcommand{\FloatTok}[1]{\textcolor[rgb]{0.00,0.00,0.81}{#1}}
\newcommand{\FunctionTok}[1]{\textcolor[rgb]{0.00,0.00,0.00}{#1}}
\newcommand{\ImportTok}[1]{#1}
\newcommand{\InformationTok}[1]{\textcolor[rgb]{0.56,0.35,0.01}{\textbf{\textit{#1}}}}
\newcommand{\KeywordTok}[1]{\textcolor[rgb]{0.13,0.29,0.53}{\textbf{#1}}}
\newcommand{\NormalTok}[1]{#1}
\newcommand{\OperatorTok}[1]{\textcolor[rgb]{0.81,0.36,0.00}{\textbf{#1}}}
\newcommand{\OtherTok}[1]{\textcolor[rgb]{0.56,0.35,0.01}{#1}}
\newcommand{\PreprocessorTok}[1]{\textcolor[rgb]{0.56,0.35,0.01}{\textit{#1}}}
\newcommand{\RegionMarkerTok}[1]{#1}
\newcommand{\SpecialCharTok}[1]{\textcolor[rgb]{0.00,0.00,0.00}{#1}}
\newcommand{\SpecialStringTok}[1]{\textcolor[rgb]{0.31,0.60,0.02}{#1}}
\newcommand{\StringTok}[1]{\textcolor[rgb]{0.31,0.60,0.02}{#1}}
\newcommand{\VariableTok}[1]{\textcolor[rgb]{0.00,0.00,0.00}{#1}}
\newcommand{\VerbatimStringTok}[1]{\textcolor[rgb]{0.31,0.60,0.02}{#1}}
\newcommand{\WarningTok}[1]{\textcolor[rgb]{0.56,0.35,0.01}{\textbf{\textit{#1}}}}
\usepackage{longtable,booktabs,array}
\usepackage{calc} % for calculating minipage widths
% Correct order of tables after \paragraph or \subparagraph
\usepackage{etoolbox}
\makeatletter
\patchcmd\longtable{\par}{\if@noskipsec\mbox{}\fi\par}{}{}
\makeatother
% Allow footnotes in longtable head/foot
\IfFileExists{footnotehyper.sty}{\usepackage{footnotehyper}}{\usepackage{footnote}}
\makesavenoteenv{longtable}
\usepackage{graphicx}
\makeatletter
\def\maxwidth{\ifdim\Gin@nat@width>\linewidth\linewidth\else\Gin@nat@width\fi}
\def\maxheight{\ifdim\Gin@nat@height>\textheight\textheight\else\Gin@nat@height\fi}
\makeatother
% Scale images if necessary, so that they will not overflow the page
% margins by default, and it is still possible to overwrite the defaults
% using explicit options in \includegraphics[width, height, ...]{}
\setkeys{Gin}{width=\maxwidth,height=\maxheight,keepaspectratio}
% Set default figure placement to htbp
\makeatletter
\def\fps@figure{htbp}
\makeatother
\setlength{\emergencystretch}{3em} % prevent overfull lines
\providecommand{\tightlist}{%
  \setlength{\itemsep}{0pt}\setlength{\parskip}{0pt}}
\setcounter{secnumdepth}{5}
\usepackage{booktabs}
\usepackage{amsthm}
\usepackage{amsmath}
\makeatletter
\def\thm@space@setup{%
  \thm@preskip=8pt plus 2pt minus 4pt
  \thm@postskip=\thm@preskip
}
\makeatother
\ifluatex
  \usepackage{selnolig}  % disable illegal ligatures
\fi
\usepackage[]{natbib}
\bibliographystyle{apalike}

\title{Master-Thesis}
\author{Axel Roth}
\date{2022-08-09}

\begin{document}
\maketitle

{
\setcounter{tocdepth}{1}
\tableofcontents
}
\hypertarget{abstract}{%
\chapter{Abstract}\label{abstract}}

Things about this thesis. why and what question should be answered. and what are the answers. (zusammenfassung)

\hypertarget{software-information-and-conventions}{%
\chapter{Software information and conventions}\label{software-information-and-conventions}}

What R and packages do i mainly use, what do i use to write this book. Mathematical conventions and why i prefere the matrix notations. And used functions.

\hypertarget{data-sources}{%
\chapter{Data Sources}\label{data-sources}}

What source do i use and what functions do exist for it

\hypertarget{activ-vs-passiv-investing}{%
\chapter{Activ vs Passiv Investing}\label{activ-vs-passiv-investing}}

The fundation of each Asset Management

passiv vs activ studie
\texttt{https://www.scirp.org/journal/paperinformation.aspx?paperid=92983}

gut gut
\url{file:///C:/Users/Axel/Desktop/Master-Thesis-All/Ziel\%20was\%20beantwortet\%20werden\%20soll/Quellen\%20nur\%20wichtige/Rasmussen2003_Book_QuantitativePortfolioOptimisat.pdf}

\hypertarget{challenges-of-passiv-investing}{%
\chapter{Challenges of Passiv Investing}\label{challenges-of-passiv-investing}}

In this Chapter we will discuss two common challenges of Passiv-Investing and create simple examples to test the PSO. The first one is the mean-variance portfolio (MVP) from the modern portfolio theory of Markowitz which is simply said an optimal allocation of assets regarding risk and return. The second challenge is the index-tracking-problem which tries to construct a portfolio which has a minimal tracking error to a given benchmark.

\hypertarget{mean-variance-portfolio-mvp}{%
\section{Mean-variance portfolio (MVP)}\label{mean-variance-portfolio-mvp}}

Markowitz has shown that diversifying the risk on multiple assets will reduce the overall risk of the portfolio. This result was the beginning of the widely used modern portfolio theorie which uses mathematical models to archive portfolios with minimal variance for a given return target. All these optimal portfolios for a given return target are called efficient and create the efficient frontier.

\hypertarget{mvp}{%
\subsection{MVP}\label{mvp}}

We have \(N\) assets and its returns on \(T\) different days which creates a return matrix \(R \in \mathbb{R}^{T \times N}\). Each element \(R_{t,i}\) contains the return of the \(i\)-th asset on day \(t\). The covariance matrix of the returns is \(\textstyle\sum \in \mathbb{R}^{N \times N}\) and the expected returns are \(\mu \in \mathbb{R}^{N}\). The MVP with risk aversion parameter \(\lambda \in [0,1]\) like shown in \citep{Mar2005} can be formalized as follows:
\begin{equation} 
\underset{w}{minimize} \ \ \ \lambda \ w^T \textstyle\sum w - (1-\lambda) \ \mu^T w
\label{eq:MVP}
\end{equation}

The risk aversion parameter \(\lambda\) defines the trade-off between risk and return. With \(\lambda = 1\), the minimization problem only contains the the variance term and so on results in a minimum variance portfolio and \(\lambda = 0\) transforms the problem to a minimization of the negative expected returns so on results in a maximum return portfolio. All possible \(\lambda \in [0, 1]\) represent the efficient frontier.

\hypertarget{mvp-example}{%
\subsection{MVP example}\label{mvp-example}}

We will analyze a small example to understand the meaning of the efficient frontier without going into detail how it was solved. First of all we are loading the daily returns of IBM, Google and Apple from the year 2020.

The cumulated daily returns are:

\includegraphics{gitbook-demo_files/figure-latex/unnamed-chunk-5-1.png}

Now we can calculate the expected daily returns and the covariance matrix for the 3 assets:

\begin{Shaded}
\begin{Highlighting}[]
\NormalTok{mu }\OtherTok{\textless{}{-}} \FunctionTok{as.vector}\NormalTok{((}\FunctionTok{last}\NormalTok{(}\FunctionTok{ret\_to\_cumret}\NormalTok{(returns))}\SpecialCharTok{/}\DecValTok{100}\NormalTok{)}\SpecialCharTok{\^{}}\NormalTok{(}\DecValTok{1}\SpecialCharTok{/}\FunctionTok{nrow}\NormalTok{(returns))}\SpecialCharTok{{-}}\DecValTok{1}\NormalTok{) }\SpecialCharTok{\%\textgreater{}\%} 
  \FunctionTok{setNames}\NormalTok{(., }\FunctionTok{colnames}\NormalTok{(returns))}
\NormalTok{mu}
\end{Highlighting}
\end{Shaded}

\begin{verbatim}
##           AAPL            IBM           GOOG 
##  0.00237641721 -0.00004622149  0.00106873786
\end{verbatim}

\begin{Shaded}
\begin{Highlighting}[]
\NormalTok{cov }\OtherTok{\textless{}{-}} \FunctionTok{as.matrix}\NormalTok{(}\FunctionTok{nearPD}\NormalTok{(}\FunctionTok{cov}\NormalTok{(returns))}\SpecialCharTok{$}\NormalTok{mat)}
\NormalTok{cov}
\end{Highlighting}
\end{Shaded}

\begin{verbatim}
##              AAPL          IBM         GOOG
## AAPL 0.0008635696 0.0004356282 0.0005337719
## IBM  0.0004356282 0.0006626219 0.0004086728
## GOOG 0.0005337719 0.0004086728 0.0005827306
\end{verbatim}

We now have all the necessary data to solve the MVP \eqref{eq:MVP} with \(\lambda \in \{0.01, 0.02, ..., 0.99, 1\}\). We calculate all 100 portfolios by solving the quadratic minimization problem for each \(\lambda\).

Following that, we convert the daily returns and standard deviation to yearly returns and standard deviation before plotting the efficient frontier.

\includegraphics{gitbook-demo_files/figure-latex/unnamed-chunk-8-1.png}

\includegraphics{gitbook-demo_files/figure-latex/unnamed-chunk-9-1.png}

\hypertarget{index-tracking-portfolio-itp}{%
\section{Index-tracking portfolio (ITP)}\label{index-tracking-portfolio-itp}}

Indices are asset baskets that are used to track the performance of a specific asset group. The well-known Standard and Poor's 500 index (short: S\&P 500), for example, tracks the top 500 stocks in the United States. All indices are not investible and only serve to visualize the performance of these asset groups without incurring transaction costs. Asset managers use such indices as benchmarks to compare the performance of their funds. Each fund has its own benchmark, which contains roughly the same assets that the manager could purchase. If the fund underperforms its benchmark, it may be an indication that the fund manager made a poor decision. That is why all fund managers strive to outperform their benchmarks through carefully chosen investments. The past has proven that this is rearly achived with activ managemnt after costs \citep{Desm2016}. This is the reason why passiv managed funds with the goal to track there benchmarks are becoming more frequent. This is why passively managed funds with the purpose of tracking their benchmarks are becoming more common. This can be accomplished through either full or sparse replication. In most circumstances, a full replication that achieves the exact performance we seek is not achievable because not all assets in an index are investable. And, if so, it would be unwise because benchmarks with numerous indexes can contain over ten thousand separate assets, resulting in a massive amount of transaction costs. A sparse replication of the performance is the most prevalent approach. To do so, the portfolio manager must define his benchmark, which should overlap with his fund's investing universe. Following that, he will reduce this universe using investor principles such as liquidity and availability. Now he can begin to optimize a portfolio, taking into account the investor constraints, in order to match the benchmark performance. Typically, this is accomplished by lowering the difference between the ITP's daily returns and the benchmark:

\[
 minimize \ \ Var(r_{p}-r_{bm})
\]

First we need to substitude \(r_{p}\) to get the portfolio weights \(w\) as follows:

\[
  r_{p} = R * w
\]

Afterwards we solve the Variance until it is displayed as a quadratic problem of \(w\):

\[
 Var(r_{p}-r_{bm}) = Var(R * w - r_{bm}) = Var(R * w) + Var(r_{bm}) - 2 \cdot Cov(R*w,r_{bm}) 
\]
Now we need to solve each of the 3 terms, startign with the simplest

\[
Var(r_{bm}) = \sigma_{bm}^2 = constant
\]

The variance of the portfolio can be solved by looking at the proof in {[}\url{http://www.math.kent.edu/~reichel/courses/monte.carlo/alt4.7d.pdf} (proof auch selber machen){]} at the linear combination of random variables section:

\[
Var(R * w) = w^T * Cov(R) * w
\]

And the last term can be solved in general as (\url{https://bookdown.org/compfinezbook/introcompfinr/Multivariate-Probability-Distributions-Using-Matrix-Algebra.html} 3.6.5):

\[
  Cov(A*a, b) = Cov(b, A*a) = E[(b-\mu_{b})(A*a-\mu_{A}*a)] = E[(b-\mu_{b})(A-\mu_{A})*a] = E[(b-\mu_{b})(A-\mu_{A})]*a = Cov(A,b) * a
\]

\begin{Shaded}
\begin{Highlighting}[]
\NormalTok{A }\OtherTok{=} \FunctionTok{matrix}\NormalTok{(}\FunctionTok{c}\NormalTok{(}\DecValTok{1}\NormalTok{,}\DecValTok{4}\NormalTok{,}\DecValTok{2}\NormalTok{,}\DecValTok{4}\NormalTok{,}\DecValTok{6}\NormalTok{,}\DecValTok{3}\NormalTok{,}\DecValTok{8}\NormalTok{,}\DecValTok{4}\NormalTok{,}\DecValTok{4}\NormalTok{,}\DecValTok{10}\NormalTok{), }\AttributeTok{ncol=}\DecValTok{2}\NormalTok{)}
\NormalTok{a }\OtherTok{=} \FunctionTok{c}\NormalTok{(}\FloatTok{0.2}\NormalTok{, }\FloatTok{0.8}\NormalTok{)}
\NormalTok{b }\OtherTok{=} \FunctionTok{c}\NormalTok{(}\DecValTok{4}\NormalTok{,}\DecValTok{4}\NormalTok{,}\DecValTok{5}\NormalTok{,}\DecValTok{5}\NormalTok{,}\DecValTok{7}\NormalTok{)}

\FunctionTok{cov}\NormalTok{(A }\SpecialCharTok{\%*\%}\NormalTok{ a, b)}
\end{Highlighting}
\end{Shaded}

\begin{verbatim}
##      [,1]
## [1,] 2.15
\end{verbatim}

\begin{Shaded}
\begin{Highlighting}[]
\FunctionTok{t}\NormalTok{(a) }\SpecialCharTok{\%*\%} \FunctionTok{cov}\NormalTok{(A, b)}
\end{Highlighting}
\end{Shaded}

\begin{verbatim}
##      [,1]
## [1,] 2.15
\end{verbatim}

\begin{Shaded}
\begin{Highlighting}[]
\FunctionTok{t}\NormalTok{(}\FunctionTok{cov}\NormalTok{(A, b)) }\SpecialCharTok{\%*\%}\NormalTok{ a }\CommentTok{\# das hier wird gebraucht}
\end{Highlighting}
\end{Shaded}

\begin{verbatim}
##      [,1]
## [1,] 2.15
\end{verbatim}

This results in the final formula of the ITP:

\begin{equation}
  \begin{split}
   Var(r_{p}-r_{bm}) & = Var(R \times w - r_{bm}) \\
   & = Var(R \times w) - 2 \cdot Cov(R \times w,r_{bm}) + Var(r_{bm})  \\
   & = w^T \times Cov(R) \times w - 2 \cdot Cov(r_{bm}, R)^T \times w + \sigma_{bm}^2
   \end{split}
   \label{eq:ITP}
\end{equation}

The minimization problem of the ITP in the general stricture which all optimizers need is:

\[
  min(\frac{1}{2} \cdot b^T \times D \times b -d^T \times b)
\]

Because it is a minimization we can ignore constant terms and stretching coefficients and still find the same minimum. This results in the general substitution of the ITP as follows:

\[
  D = Cov(R)
\]

and

\[
d = Cov(r_{bm}, R)
\]

Now we need to add some basic constraints like in the MVP, to sum up the weights to 1 and being long only.

\hypertarget{example-itp}{%
\subsection{Example ITP}\label{example-itp}}

We will show the results of tracking the S\&P 500 with a tracking portfolio that can only invest in IBM, Apple and Google without going into details:

\begin{verbatim}
##      AAPL       IBM      GOOG 
## 0.2677928 0.4041880 0.3280192
\end{verbatim}

\includegraphics{gitbook-demo_files/figure-latex/unnamed-chunk-12-1.png}

\hypertarget{analytical_solver}{%
\chapter{Analytical\_Solver}\label{analytical_solver}}

example to solve problems with analytical solvers

\hypertarget{simple_particle_swarm_optimization}{%
\chapter{Simple\_Particle\_Swarm\_Optimization}\label{simple_particle_swarm_optimization}}

first pso examples and explainations

  \bibliography{book.bib,packages.bib}

\end{document}
